\documentclass[12pt,a4paper]{article}
\usepackage[utf8]{inputenc}
\usepackage{amsmath}
\usepackage{amsthm}
\usepackage{amsfonts}
\usepackage{amssymb}
\usepackage[overload]{empheq}
\usepackage{fullpage}
\usepackage[linktocpage]{hyperref}
\usepackage{titlesec}
\newcommand{\sectionbreak}{\clearpage}

\title{Méthodes de résolution approchées du T.S.P.}
\author{Dorian Dumez \and Jocelin Cailloux}

\begin{document}

\maketitle
\tableofcontents

\section{Heuristiques de constructions}
\subsection{N.N.H.}
Cet algorithme est une heuristique de construction gloutonne qui, partant d'une ville donnée, construit le cycle hamiltonien de proche en proche. A partir d'une ville d'origine, l'algorithme sélectionne comme prochaine ville celle qui est la plus proche de la ville courante.\\

Nous remarquons tout de suite l'impact du choix de la ville de départ sur la solution, mais estimer la qualité de la solution en fonction de chaque ville est compliqué, nous choisissons donc le point de départ de manière arbitraire. En effet, nous sélectionnons toujours comme point de départ la première ville située dans le tableau de notre distancier.\\

Notre distancier représentant le graphe sous forme d'une matrice d'adjacence, nous disposons toujours de la distance entre deux ville en temps constant. Cet algorithme énumère pour chaque ville (nous parcourons chaque ville une et une seule fois étant donné la construction d'un cycle hamiltonien), toutes les villes non déjà parcourues pour trouver la plus proche. Nous déduisons que cet algorithme a une complexité temporelle en $O(nbVille^2)$. De plus sa complexité spatiale est en $O(nbVille)$, en effet, il n’alloue que le parcours correspondant à la permutation identité avant de le modifier sur place.\\

\subsection{R.G.S.C}
Nous avons choisi d'implémenter une heuristique de construction de notre invention, en effet, nous voulions essayer quelque chose d'original. Nous avons donc imaginé un algorithme qui se base sur celui de Gale et Shapley qui résout le problème du mariage stable. Nous l'avons donc appelé ainsi, Recursive Gale-Shapley Circuit.\\

Nous avons pensé qu'il est pertinent de construire des couples de villes de telle manière qu'aucune puisse s'unir avec une autre ville sans augmenter la longueur totale des distances entre les couples. L'idée est donc de former de tels couples de manière récursive, une fois que les villes sont unies deux à deux, nous réitérons l'algorithme afin d'unir les couples précédement formés deux à deux de la même manière.\\

Chaque ville (puis chaque sous-couple par la suite), possède une liste de préférences des autres couples triés par ordre décroissant des distances. Nous avons supposé que la complexité de cet algorithme serait du même ordre que celui de Gale-Shapley multiplié par log(n), log(n) étant le nombre d'itérations effecutées. L'algorithme de Gale-Shapley s'effectue en $O(n^2)$, nous avons donc pour notre algorithme RGSC une complexité de $O(log(nbVille) \times nbVille ^2)$.\\

Après avoir implémenté la méthode, nous avons pu constater que celle-ci ne converge pas. En effet, pour certaines instances, quelques groupes ne pouvaient être reliés, ils perdaient leur liaison au dépit d'autres groupes. Ce phénomène ne survient pas dans le cadre d'utilisation de l'algorithme du mariage stable (deux groupes distincts avec des unions entre les éléments de chaque groupe). Nous avons donc choisi de modifier cet algorithme pour le faire converger. Ainsi, une fois relié, un groupe ne cherche pas à se relier à un autre groupe plus proche. Seuls les groupes non reliés cherche à se relier avec l'autre groupe le plus proche et peuvent casser une laison si la leur est plus favorable. Avec cette méthode, nous avons pu faire converger l'algorithme, de plus, nous obtenons de meilleurs résultats. Cependant, l'heuristique de construction NNH reste plus efficace.\\
\section{Opérateurs de modification}

\subsection{2-opt}
Cet opérateur consiste à sélectionner deux arcs (i.e. deux paires de villes adjacentes dans le parcours) et à tester si leur croisement fournit une meilleure solution. Si nous posons iD, iF les extrémités du premier arc et jD, jF celle du deuxième arc, nous entendons par croiser les arcs allers de iD à jD puis de iF à jF au lieu de iD à iF puis de jD à jF comme c’était le cas dans la solution. Graphiquement nous observons que c'est en réalité l'inverse qui améliore la solution, en effet si les deux arcs s’intersectent dans le plan (d'une carte où les villes sont positionnées dans un repère par exemple) alors les inverser fait disparaître cette intersection et réduit la taille du chemin parcouru.\\

Il est alors évident que les 4 villes (iD, iF, jD, jF) doivent êtres différentes, autrement, nous ne ferions que modifier l'ordre de parcours de ces villes (on ne fait qu'inverser une permutation) mais le T.S.P. sur lequel nous travaillons est symétrique donc cela ne change rien. Il est aussi évident que nous considèrons des paires d'arcs et non des couples car cela reviendrait à tester deux fois le même changement.\\

Nous avons implémenté cet opérateur sous deux formes : en descente et en plus profonde descente.
En plus profonde descente on teste toute les paires d'arc et on applique seulement celle qui conduit à la meilleure solution, ici celle dont le parcours est le plus cour.\\

En descente on teste aussi toutes le paires d'arc mais cette fois ci on applique la modification dès qu'elle améliore la solution. L'algorithme ne s’arrête pas à la première solution trouvée pour éviter de s'appesantir sur le début du cycle et pour éviter de le re-tester un trop grand nombre de fois à partir du moment où il se serait stabilisé.\\

Dans les deux cas nous testons $O(nbVille^2)$ changement mais appliquer l'opérateur 2-opt s’effectue en $O(nbVille)$ car nous devons inverser le sens de parcours des villes situées entre iF et jD. Donc l'algorithme de descente est en $O(nbVille^3)$ tandis que celui de plus profonde descente est en $O(nbVille^2)$. Mais cette complexité n'est que celle de amelioreSol2opt et amelioreSol2optPPD, donc ne prend pas en compte la vitesse de convergence de la solution vers un minimum local.\\
Ces complexités sont permises grâce l'obtention en avance et en temps constant la taille du cycle hamiltonien après la modification.
 
\subsection{3-opt}
Cet opérateur est une généralisation du 2-opt avec 3 arc. Tout les mouvements permis par le 2-opt le sont donc aussi par le 3-opt. Donc contrairement au 2-opt qui n'offre qu'une seule possibilité de modification, le 3-opt en offre 7 dont certaines ne peuvent pas être réalisées par une succession de 2-opt.\\

Cet opérateur a aussi été implémenté en descente et en plus profonde descente.\\
Pour les mêmes raisons que pour le 2-opt, dans l'algorithme de descente nous testons tous les triplet possibles à chaque fois au plutôt que de s’arrêter au premier mouvement améliorant trouvé.\\

Dans les deux cas nous testons tous les triplets d'arcs possibles, il y en a $O(nbVille ^3)$. De même que pour le 2-opt appliqué, une modification s'effectue en $O(nbVille)$ car des inversions du sens de parcours sont parfois nécessaires. Donc selon le même principe que pour le 2-opt l'opérateur 3-opt est en $O(nbVille^4)$ pour la version classique et en $O(nbVille^3)$ pour la plus profonde descente. Mais cela ne prend pas en considération la vitesse de convergence vers un minimum local.\\

\section{Algorithmes de recherche locale}

Les deux opérateurs ayant une version normale et une en plus profonde descente tous les algorithmes suivants ont aussi deux versions. Ce sont toujours les mêmes, à la différence que l'une utilisera les opérateurs classiques et l'autre ceux en plus profonde descente.

\subsection{Descente avec 2-opt}

Cet algorithme part d'une solution que lui fournit une heuristique de construction puis l’améliore uniquement à l'aide de l'opérateur 2-opt. On utilise toujours le même opérateur sur la solution sur laquelle on travaille. On le fait jusqu’à tomber dans un minimum local, c'est à dire jusqu’à que 2-opt ne puisse plus améliorer la solution.

\subsection{Descente avec 3-opt}

Cet algorithme est identique au précédent mais avec l'opérateur 3-opt.

\subsection{VND}

Cette fonction implémente un algorithme V.N.D utilisant les opérateurs 2 et 3 opt dans cet ordre. Elle part d'une solution fournie par une heuristique de construction. Elle utilise le critère d’arrêt standard du VND qui est d'avoir trouvé un minimum local, c'est à dire que ni 2-opt ni 3-opt ne peuvent améliorer la solution.

\subsection{VNS}

Cette fonction implémente un algorithme V.N.S utilisant les opérateurs 2 et 3 opt dans cet ordre. Elle part d'une solution fournie par une heuristique de construction. Elle va s’arrêter quand elle aura utilisé, successivement, une fois la recherche de voisin puis l'amélioration, avec 2-opt puis avec 3-opt sans amélioration. Il faudrait donc mener de plus ample expérimentation concernant ce critère d’arrêt. Notamment en répétant ces 2 recherches plusieurs fois avant d'abandonner.

\section{Expérimentation}
\subsection{En partant de N.N.H.}
PPD désigne la variante de l'algorithme utilisant la plus profonde descente.\\
Tout ce qui concerne le VNS dans tableau \ref{NNHpourcentageperf} et \ref{NNHtemps} sont des moyennes effectuées sur 10 exécutions. Ces dernières sont données avec plus de détails dans le tableau \ref{variationVNSNNH}.\\

\begin{table}[!h]
\leftskip -1cm
{
\begin{tabular}{|*{10}{c|}}
  \hline
  nbVille & N.N.H. & 2-opt & 3-opt & 2-opt PPD & 3-opt PPD & VND & VND PPD & VNS & VNS PPD \\
  \hline
  48 & 281 & 225 & 218 & 223 & 223 & 218 & 223 & 219 & 222 \\
  52 & 19.0 & 6.80 & 5.96 & 3.96 & 3.40 & 2.56 & 3.41 & 4.23 & 3.41 \\
  130 & 23.9 & 10.3 & 4.01 & 5.98 & 1.77 & 4.57 & 4.62 & 3.59 & 4.61 \\
  150 & 25.5 & 3.88 & 3.39 & 1.45 & 1.31 & 1.85 & 0.80 & 3.38 & 0.801 \\
  280 & 23.3 & 10.7 & 6.86 & 7.63 & 3.04 & 6.39 & 6.64 & 5.70 & 6.62 \\
  \hline
\end{tabular}
}
\caption{distance supplémentaire en pourcentage par rapport à la solution optimale}
\label{NNHpourcentageperf}
\end{table}

\begin{table}[!h]
\leftskip -1.4cm
{
\begin{tabular}{|*{10}{c|}}
  \hline
  nbVille & N.N.H. & 2-opt & 3-opt & 2-opt PPD & 3-opt PPD & VND & VND PPD & VNS & VNS PPD \\
  \hline
  48 & 1e-05 & 0.0002 & 0.0151 & 0.0004 & 0.0380 & 0.0116 & 0.0042 & 0.0101 & 0.0821 \\
  52 & 1e-05 & 0.0002 & 0.0165 & 0.0010 & 0.0573 & 0.0102 & 0.0110 & 0.0143 & 0.0193 \\
  130 & 5e-05 & 0.0020 & 0.2013 & 0.0043 & 1.4373 & 0.1155 & 0.2935 & 0.3854 & 0.6382 \\
  150 & 0.0001 & 0.0032 & 0.2481 & 0.0055 & 1.4498 & 0.1817 & 0.1885 & 0.4651 & 0.5097 \\
  280 & 0.0004 & 0.0122 & 2.0138 & 0.0351 & 19.013 & 1.2042 & 3.0444 & 3.695 & 5.318 \\ 
  \hline
\end{tabular}
}
\caption{temps d’exécutions en secondes}
\label{NNHtemps}
\end{table}

\begin{table}[!h]
\centering
\begin{tabular}{|*{5}{c|}}
  \hline
  nbVille & variation temps & variation résultat & variation temps & variation résultat \\
  ~ & VNS & VNS & VNS PPD & VNS PP \\
  \hline
  48 & 398 & 1.57 & 167 & 0.750 \\ 
  52 & 286 & 7.62 & 67.7 & 2e-14 \\
  130 & 247 & 4.17 & 83.1 & 0.0305 \\
  150 & 75.3 & 5.92 & 44.5 & 4e-14 \\
  280 & 334 & 5.91 & 25.2 & 0.183 \\
  \hline
\end{tabular}
\caption{variation proportionnelle des algorithmes VNS}
\label{variationVNSNNH}
\end{table}

De ces première expérimentation on peut déduire que :
\begin{itemize}
\item
2-opt est très clairement le pus rapide, malheureusement cela se ressent sur la qualité de ses solutions.
\item
la version plus profonde descente augmente bien la qualité des solutions (surtout pour 2 et 3-opt seul) mais cela se ressent fortement sur le temps d’exécution.
\item
3-opt seul ne semble présenter aucun intérêt car vnd le surpasse en tout point. Mais il sera tout de même gardé pour la suite car il pourrai réagir différemment face à la diversité de solution proposé par GRASP.
\item
3-opt PPD seul propose les solutions de meilleures qualité, malheureusement son utilisation est très compromise par le temps de calcul qu'il requiert. Au contraire 2-opt PPD propose de bonnes solutions, sans être aussi bonne, mais sont temps d’exécution est des plus compétitif.
\item
Quand a vnd et vns il propose des solutions de qualités, légèrement meilleures que celles de 2-opt PPD, mais son plus lent que ce dernier. Et entre eux ils sont difficilement comparable car la comparaison dépend des instances.
\item
VND est relativement stable du point de vue de la qualité des solutions mais ne l'est pas du tout selon le temps d’exécution.
\item
VNS est déjà nettement plus stable du coté des temps d’exécutions et renforce encore plus cette stabilité au regard de la qualité des solutions.
\end{itemize}

\subsection{En partant de R.G.S.C}
Nous avons rencontré plus de difficultés que prévu lors de l'implémentation de cet algorithme. En dehors de l'aspect programmation et déboguage qui étaient particulièrement compliqué, nous avons remarqué que celui-ci ne converge pas pour l'itération de 280 villes que nous avons.\\
Nous n'avons pas encore déterminé s'il s'agît d'une erreur de programmation ou bien de l'algorithme qui ne converge pas. En effet, l'algorithme de Gale-Shapley permet de créer un mariage stable entre deux ensembles différents, ici, nous marions les éléments d'un ensemble avec des élements de ce même ensemble. Cette différence fondamentale qui ne semble pas avoir d'impact intuitivement, pourrait en effet empêcher la convergence de l'algorithme.\\
Nous ne présentons donc pas de statistiques avec cet algorithme pour la dernière instance, mais nous présentons les autres instances, pour lesquelles l'algorithme converge.\\
\begin{table}[!h]
\leftskip -1.2cm
{
\begin{tabular}{|*{10}{c|}}
  \hline
  nbVille & R.G.S.C. & 2-opt & 3-opt & 2-opt PPD & 3-opt PPD & VND & VND PPD & VNS & VNS PPD \\
  \hline
  48 & 357 & 220 & 221 & 225 & 222 & 220 & 218 & 221 & 216 \\
  52 & 24.1 & 1.55 & 3.91 & 2.56 & 2.56 & 1.55 & 2.56 & 2.37 & 1.80 \\
  130 & 38.5 & 5.14 & 2.98 & 2.83 & 0.845 & 1.60 & 1.43 & 2.92 & 1.49 \\
  150 & 38.8 & 6.68 & 4.43 & 3,95 & 1.26 & 4.89 & 3.48 & 2.62 & 3.48 \\
  \hline
\end{tabular}
}
\caption{distance supplémentaire en pourcentage par rapport à la solution optimale}
\label{RGSCpourcentageperf}
\end{table}

\begin{table}[!h]
\leftskip -1.6cm
{
\begin{tabular}{|*{10}{c|}}
  \hline
  nbVille & R.G.S.C. & 2opt & 3opt & 2opt PPD & 3opt PPD & VND & VND PPD & VNS & VNS PPD \\
  \hline
  48 & 0.0132 & 0.0003 & 0.0132 & 0.0010 & 0.0650 & 0.0041 & 0.0119 & 0.0239 & 0.0844 \\
  52 & 0.0155 & 0.0003 & 0.0135 & 0.0006 & 0.0382 & 0.0048 & 0.0052 & 0.0091 & 0.0887 \\
  130 & 0.1389 & 0.0015 & 0.1169 & 0.0071 & 1.620 & 0.1194 & 0.2361 & 0.2992 & 0.5526 \\
  150 & 0.2207 & 0.0019 & 0.1716 & 0.0087 & 2.593 & 0.1735 & 0.1914 & 0.5171 & 0.5229 \\
  \hline
\end{tabular}
}
\caption{temps d’exécution en secondes}
\label{RGSCtemps}
\end{table}

\begin{table}[!h]
\centering
\begin{tabular}{|*{5}{c|}}
  \hline
  nbVille & variation temps & variation résultat & variation temps & variation résultat \\
  ~ & VNS & VNS & VNS PPD & VNS PP \\
  \hline
  48 & 219 & 2.81 & 106 & 0.548 \\ 
  52 & 302 & 3.69 & 105 & 2.53 \\
  130 & 109 & 2.42 & 36.8 & 0.641 \\
  150 & 177 & 5.43 & 63.3 & 7e-14 \\
  \hline
\end{tabular}
\caption{variation proportionnelle des algorithmes VNS}
\label{variationVNSRGSC}
\end{table}

On remarque alors que les différents opérateurs conservent le même rapport entre eux en partant de solutions construites par RGSC. Mais même après amélioration cet algorithme de construction est totalement dominé par NNH.\\

\subsection{Modification du critère d’arrêt de VNS}

Dans cette section je détaille les expérimentations faites sur VNS et VNS-PPD. En effet dans les tableaux précédent on s'arrête dès que 2-opt puis 3-opt n'ont pas amélioré successivement. Mais leurs laisser plusieurs essais. Donc on effectue la boucle 2-opt - 3-opt k fois, on relève la moyenne de la différence proportionnelle de valeur et la moyenne de temps d’exécution. Toutes ces exécutions sont effectués en partant de solution construite avec NNH.

\begin{table}[!h]
\centering
\begin{tabular}{|*{7}{c|}}
  \hline
  nbVille & VNS -1 & VNS PPD -1 & VNS -2 & VNS PPD -2 & VNS -3 & VNS PPD -3 \\
  \hline
  48 & 219 & 223 & 219 & 223 & 217 & 223 \\ 
  52 & 4.48 & 3.41 & 2.98 & 3.66 & 2.48 & 3.40 \\
  130 & 3.26 & 4.62 & 2.03 & 4.62 & 1.85 & 4.53 \\
  150 & 3.28 & 0.80 & 2.43 & 0.80 & 1.86 & 0.99 \\ 
  280 & 4.64 & 6.44 & 3.53 & 6.53 & 3.10 & 6.59 \\
  \hline
\end{tabular}
\caption{modification du critère d’arrêt de VNS : valeurs}
\label{stopingRuleVNSval}
\end{table}

\begin{table}[!h]
\centering
\begin{tabular}{|*{7}{c|}}
  \hline
  nbVille & VNS -1 & VNS PPD -1 & VNS -2 & VNS PPD -2 & VNS -3 & VNS PPD -3 \\
  \hline
  48 & 0.0177 & 0.0114 & 0.0281 & 0.0123 & 0.0412 & 0.0162 \\ 
  52 & 0.0167 & 0.0194 & 0.0322 & 0.0271 & 0.0484 & 0.0368 \\
  130 & 0.4620 & 0.6899 & 0.7016 & 0.8776 & 1.200 & 1.029 \\
  150 & 0.4204 & 0.5822 & 0.9710 & 0.7370 & 1.445 & 1.007 \\  
  280 & 4.587 & 5.923 & 10.63 & 8.204 & 12.34 & 9.188 \\
  \hline
\end{tabular}
\caption{modification du critère d’arrêt de VNS : temps}
\label{stopingRuleVNStemps}
\end{table}

Premièrement cette modification dessert complètement la version plus profonde descente. La qualité de ses solutions n'est même pas forcement augmenté (logiquement elle devrai toujours l’être mais l’aléatoire fait qu'elle ne l'est même pas forcement) tellement l’amélioration est faible. Par contre l'impact sur le temps d’exécution est très clairement visible.\\

Sur la version classique le bilan est plus mitigé. En effet cette modification viens grandement améliorer la qualité des solutions, surtout le passage de 1 à 2 tours. Mais l'opérateur part alors en rapidité : on a en moyenne un facteur 2 entre le temps d’exécution entre 1 et 2 tours.\\ 

\section{GRASP}

\subsection{Construction des solutions}

GRASP construit ses solutions avec un algorithme de NNH modifié pour include de l'aléatoire. NNH à été choisis car il propose de meilleurs résultats que RGSC, et nos recherches locales se comportent aussi mieux dessus. De plus RGSC ne convergent pas dans tous les cas il aurait dangereux de l'utiliser de manière aussi intensive.\\

La version classique de NNH part d'une ville, parcours toutes les autres pour trouver la plus proche, y va et recommence.  Etant donné que pour chaque ville on doit parcourir toutes celles qui n'ont pas encore été visité cet algorithme est en $O(nbVille ^2)$.\\

Dans cette version l'algorithme prend en paramètre "le taux d'aléatoire" $\alpha$. On va encore parcourir toutes les villes en allant ensuite vers une ville proche. Pour décider vers quelle ville on va se diriger on parcours toutes les villes qui n'ont pas encore été visité pour trouver la plus proche et la plus loin. A partir de cela, et de $\alpha$, on détermine la distance maximale acceptable. On se dirige alors vers une ville tiré aléatoirement dans l'ensemble de celles qui ne se situe pas trop loin, selon cette distance critique. Même si cet version est un peu plus lourde elle reste en $O(nbVille ^2)$.\\

\subsection{Premières expérimentations}

On expérimente l'algorithme GRASP avec plusieurs algorithme de recherche locale. Le critère d’arrêt choisis est un nogood réglé empiriquement, et de manière constante, sur le nombre de ville divisé par 5.\\
Les expérimentations sont, comme toujours, faites sur 10 expérimentations. Mais quand ce temps était trop grand une seule expérimentation a été faite. Ces cas sont noté d'une étoile. Les exécutions vraiment trop longue ont été arrêtés et noté par un tiret.\\
VNS à été testé avec 1 et 2 tours, tandis que sa version plus profonde descente n'a été utilisé qu'avec 1 tours.\\

\begin{table}[!h]
\leftskip -0.7cm
{
\begin{tabular}{|*{10}{c|}}
  \hline
  nbVille & 2opt & 3opt & 2optPPD & 3optPPD & VND & VNDPPD & VNS-1 & VNS-2 & VNSPPD \\
  \hline
  48 & 217 & 217 & 218 & 216 & 216 & 216 & 215 & 215 & 216 \\ 
  52 & 2.56 & 1.44 & 2.01 & 0.77 & 1.83 & 1.64 & 0.67 & 0.10 & 0.94 \\
  130 & 4.10 & 1.63 & 2.99 & 0.66* & 1.67 & 1.38 & 1.13 & 0.74 & 1.57 \\
  150 & 1.85 & 1.71 & 1.52 &  0.75* & 1.84 & 1.03 & 0.91 & $0.86^1$ & 1.02 \\  
  280 & 6.32 & 3.10* & 6.01 & - & 3.06* & - & 1.99* & 1.67* & 3.51* \\
  \hline
\end{tabular}
}
\caption{différence de valeur de GRASP en fonction de la recherche locale}
\label{recherchelocaleGRASPvaleurs}
\end{table}

\begin{table}[!h]
\leftskip -1.1cm
{
\begin{tabular}{|*{10}{c|}}
  \hline
  nbVille & 2opt & 3opt & 2optPPD & 3optPPD & VND & VNDPPD & VNS-1 & VNS-2 & VNSPPD \\
  \hline
  48 & 0.0146 & 0.0917 & 0.0169 & 0.9165 & 0.0886 & 0.1262 & 0.1924 & 0.3626 & 0.3424 \\ 
  52 & 0.0161 & 0.1231 & 0.0201 & 1.511 & 0.0779 & 0.1326 & 0.3465 & 0.5166 & 0.3408 \\
  130 & 0.1590 & 5.880 & 0.8385 & 249* & 6.660 & 14.9 & 17.46 & 38.5 & 30.8 \\
  150 & 0.2312 & 8.868 & 1.364 & 309* & 6.82 & 30.9 & 33.68 & $39.23^1$ & 66.1 \\  
  280 & 1.653 & 124* & 17.02 & - & 95* & - & 397* & 509* & 1103* \\
  \hline
\end{tabular}
}
\caption{temps de GRASP en fonction de la recherche locale}
\label{recherchelocaleGRASPtemps}
\end{table}
$~^1$ la huitième exécution sur les 10 à été arrêté au bout de plusieurs minutes.\\

On remarque alors que des recherches locales comme le 3-opt PPD ou le VNS-2 permettent d'obtenir des solutions d'une grande qualité. Malheureusement ces versions passent mal à l’échelle à cause de leurs temps d’exécution. Donc si l'on souhaite traiter efficacement de grosse instance le 2-opt est obligatoire.\\

Mais dans tous les cas on remarque que l'utilisation en multistart de ces différentes heuristiques permet d'obtenir des solutions de bien meilleure qualité que lorsque elles ne sont appelé qu'une seule fois.\\

\subsection{Arrêt probabiliste}

C.C. Ribeiro, I. Rosseti et R.C. Souza ont montré dans leur papier "effective probabilistic stopping rules for randomized metaheuristics: GRASP implementations" que la qualité d'une solution généré et amélioré par GRASP suit une loi normale. Il est donc possible d'utiliser cela comme condition d’arrêt de GRASP. Au regard des solution déjà généré on peut estimer les paramètre de cette loi normale, et l'on décide de s’arrêter si la probabilité d'avoir une meilleure solution devient trop faible.\\

Une loi normale est totalement décrite par sa moyenne $\mu$ et son écart-type $\sigma$. Pour estimer $\mu$ nous avons choisis l’estimateur des moments d'ordre 1 (simple et efficace) : $\widehat{\mu} = \frac{1}{n} \sum \limits_{k = 1} ^{n} X_k$. Pour estimer $\sigma$ nous avons utilisé la variance empirique corrigé (l'estimateur des moments d'ordre 2 avec un facteur correctif pour qu'il soit sans biais) : $\widehat{\sigma}^2 = \frac{1}{n-1} \sum \limits_{k = 1} ^{n} \left( X_k - \bar{X}_k \right) ^2$.\\
Mais pour éviter de refaire des calculs et d'avoir à mémoriser toutes les valeurs obtenues nous avons utilisé une forme développé de cette dernière : $\widehat{\sigma}^2 = \frac{1}{n-1} \sum \limits_{k=0} ^n \left( X_k ^2 \right) - \frac{ \bar{X}_k }{n-1} \sum \limits_{k=1} ^{n} \left( X_k \right)$ qui se calcule aisément avec des accumulateurs. En effet si note $sumX_k = \sum \limits_{k=1} ^{n} X_k$ et $sumX2_{k} = \sum \limits_{k=1} ^{n} X_k ^2$ alors $\widehat{\sigma}^2 = \frac{sumX2_k}{n-1} - \frac{sumX_k ^2}{n(n-1)}$. Mais pour que ces formules soient valides il faut que $n > 1$ donc on force toujours le premier tour de boucle.\\

Si l'on souhaite s’arrêter dès que la probabilité de trouver une meilleure solution est inférieur à $\alpha$ il faut calculer cette dernière. La probabilité d'avoir une solution d'une taille de moins de $x$ une est égal à $F_{N(\mu,\sigma^2)} (x) = \int _{- \infty} ^{x} \frac{1}{\sqrt{2 \pi \sigma ^2}} e^{ \frac{\left( x - \mu \right)^2}{2 \sigma^2} } dx$. Or on ne peux exprimer cela sans intégrale. Mais on certaines valeurs approchées de $F_{N(0,1)}$ :
\begin{itemize}
\item
$F_{N(0,1)} (-2.06) \leqslant 0.02$
\item
$F_{N(0,1)} (-1.65) \leqslant 0.05$
\item
$F_{N(0,1)} (-1.48) \leqslant 0.07$
\item
$F_{N(0,1)} (-1.29) \leqslant 0.1$
\end{itemize}
De plus on sait que si $X \sim N(0,1)$ et que $Y = \mu + \sigma X$ alors $Y \sim N(\mu, \sigma ^2)$ donc $F_{N(\mu,\sigma^2)} (y) = F_{N(0 , 1)} (\frac{y-\mu}{\sigma})$. Donc on peut dire que si $y \leqslant \mu + \sigma F^{-1} _{N(0,1)} ( \alpha)$ alors la probabilité d'avoir une meilleur solution est inférieure à $\alpha$.\\

En pratique on ne connais bien évidement ni $\mu$ ni $\sigma$ donc on utilise $\widehat{\mu}$ et $\widehat{\sigma}$ décrit précédemment. Et pour les valeurs de $\alpha$, on les choisis à l'avance pour fixer la valeur de $F^{-1} _{N(0,1)} ( \alpha)$ à utiliser, en effet $\alpha$ n’appairait plus.\\

\subsection{Nouvelles expérimentations}

Au vu de cette nouvelle idée il faut désormais refaire des tests. Mais on se restreindra cette fois aux heuristiques d'amélioration les plus prometteuses :
\begin{itemize}
\item
2-opt pour sa rapidité
\item
VNS-2 pour sa qualité de solution, sans être aussi lent que VNS-ppd
\item
VND comme solution intermédiaire qui pourrait se révéler intéressante
\end{itemize}

Notre algorithme GRASP commence toujours par construire une solution avec un NNH déterministe avant de l’améliorer avec VND. Le but étant d'avoir une bonne solution dès le début. Mais lorsque GRASP utilise l’opérateur 2-opt on se rend compte qu'une part importante de son temps d’exécution est due à cette première recherche. En effet VND avait été choisis pour cette tache car il produit de bonnes solutions mais il n'est pas trop lent (surtout qu'il n'est utilisé qu'une seule fois). Mais étant donné la vitesse de GRASP avec 2-opt on va remplacer, dans les exécutions avec 2-opt, cette amélioration par une descente avec 2-opt.

\begin{table}[!h]
\centering
\begin{tabular}{|*{6}{c|}}
  \hline
  nbVille & 2\% & 5\% & 7\% & 10\% & nogood \\
  \hline
  48 & 216 & 217 & 218 & 219 & 217 \\
  52 & 1.34 & 3.03 & 3.47 & 4.61 & 2.98 \\
  130 & 2.87 & 3.51 & 4.49 & 4.57 & 3.67 \\
  150 & 3.38 & 3.68 & 3.71 & 3.73 & 3.75 \\
  280 & 7.33 & 7.81 & 8.15 & 8.27 & 6.64 \\
  \hline
\end{tabular}
\caption{différence de valeur de GRASP en fonction de $\alpha$ avec 2-opt}
\label{recherchelocaleGRASPval2optproba}
\end{table}


\begin{table}[!h]
\centering
\begin{tabular}{|*{6}{c|}}
  \hline
  nbVille & 2\% & 5\% & 7\% & 10\% & nogood \\
  \hline
  48 & 0.1599 & 0.0105 & 0.0070 & 0.0051 & 0.0033 \\ 
  52 & 0.0250 & 0.0137 & 0.0046 & 0.0042 & 0.0041 \\
  130 & 0.2038 & 0.0594 & 0.0344 & 0.0278 & 0.0743 \\
  150 & 0.1390 & 0.0501 & 0.0523 & 0.0293 & 0.0727 \\  
  280 & 0.4688 & 0.2335 & 0.1735 & 0.1181 & 0.8435 \\
  \hline
\end{tabular}
\caption{temps de GRASP en fonction de $\alpha$ avec 2-opt}
\label{recherchelocaleGRASPtemps2optproba}
\end{table}

Pour VND et VNS-2 on garde l'initialisation avec VND, car VNS-2 et VND s’exécutent dans des temps semblables.

\begin{table}[!h]
\centering
\begin{tabular}{|*{6}{c|}}
  \hline
  nbVille & 2\% & 5\% & 7\% & 10\% & nogood \\
  \hline
  48 & - & 216 & 216 & 217 & 216 \\
  52 & 0.47 & 0.95 & 1.28 & 1.87 & 1.83 \\
  130 & 0.61* & 0.95 & 2.11 & 1.98 & 1.67 \\
  150 & 1.21* & 1.81 & 1.83 & 1.85 & 1.84 \\
  280 & 2.90* & 3.06* & 3.06* & 3.06* & 3.06* \\
  \hline
\end{tabular}
\caption{différence de valeur de GRASP en fonction de $\alpha$ avec VND}
\label{recherchelocaleGRASPtempsVNDproba}
\end{table}

\begin{table}[!h]
\centering
\begin{tabular}{|*{6}{c|}}
  \hline
  nbVille & 2\% & 5\% & 7\% & 10\% & nogood \\
  \hline
  48 & - & 0.4871 & 0.2548 & 0.0920 & 0.0886 \\
  52 & 0.3417 & 0.1380 & 0.1140 & 0.0929 & 0.0779 \\
  130 & 132.5* & 0.1395 & 2.902 & 2.679 & 6.660 \\
  150 & 116* & 10.54 & 4.295 & 2.443 & 6.82 \\
  280 & 220.8 & 118* & 109.2* & 111* & 95* \\
  \hline
\end{tabular}
\caption{temps de GRASP en fonction de $\alpha$ avec VND}
\label{recherchelocaleGRASPvalVNSproba}
\end{table}

La combinaison de l’opérateur 2-opt avec l'arret probabiliste produit de bon résultats. On remarque en effet qu'avec $\alpha = 0.02$ on obtiens généralement de meilleures solution, donc on contrebalance le défaut de 2-opt, sans que cela ne prenne trop de temps, donc sans perdre son avantage.\\

Avec VNS de nombreux tests ont été effectué, mais ils sont tellement mauvais que cette possibilité a été abandonné. Et avec VND ils ne sont pas probant non plus. Donc on peut désormais exclure cette combinaison : VND et VNS-2 offre de très belle qualité de solution mais fonctionne beaucoup mieux avec le critère nogood.\\
Et on pourrai penser à associer nogood au critère probabiliste pour éviter la non terminaison (si la variance est trop importante le critère d’arrêt deviens impossible car requiert des solutions de meilleures qualité que l'optimum). Mais cela ne pressente aucun intérêt car le problème surviens quand le critère deviens impossible mais la force de ce critère est d'accepter d'attendre un peu plus longtemps en attendent une bonne solution, donc il serai coupé par le nogood avant qu'elle n’apparaisse.\\

\subsection{Stabilité des meilleures possibilités}

Maintenant que l'on a dégagé des paramétrages de  GRASP qui semble le mieux fonctionner :
\begin{itemize}
\item
2-opt partant de 2-opt et arrêté par critère probabiliste à 2\%
\item
VND partant de VND et arrêté par nogood
\item
VNS-1 partant de VND et arrêté par nogood. On ne prend pas VNS-2 car le critère d’arrêt probabiliste ne permet pas de réduire son temps d’exécution. Donc VNS-1 donne des solutions de légèrement moins bonne qualité mais en un temps plus raisonnable.
\end{itemize}
Pour conclure on va étudier la stabilité, de la qualité des solution et du temps d’exécution, de ces algorithmes.\\

\begin{table}[!h]
\centering
\begin{tabular}{|*{4}{c|}}
  \hline
  nbVille & 2-opt & VND & VNS-1 \\
  \hline
  48 & 20779 & 145.3 & 263.1 \\
  52 & 2270 & 143.0 & 165.7 \\
  130 & 1226 & 62.86 & 149.6 \\
  150 & 2257 & 59.68 & 75.76 \\
  280 & 1912 & 136.1 & - \\
  \hline
\end{tabular}
\caption{Variation proportionnelle du temps d’exécution}
\label{varsolGRASP}
\end{table}

\begin{table}[!h]
\centering
\begin{tabular}{|*{4}{c|}}
  \hline
  nbVille & 2-opt & VND & VNS-1 \\
  \hline
  48 & 0.529 & 1.03 & 0.404 \\
  52 & 3.32 & 2.53 & 2.53 \\
  130 & 3.12 & 0.691 & 1.02 \\
  150 & 1.03 & 0.064 & 1.14 \\
  280 & 3.30 & 0.951 & - \\
  \hline
\end{tabular}
\caption{Variation proportionnelle de la qualité des solutions}
\label{vartempsGRASP}
\end{table}

On regrette donc le manque de stabilité du point de vue temporel des algorithme, surtout celui de 2-opt. Mais pour ce dernier cette instabilité n'est pas trop dérangeante car il reste rapide même dans ses pires exécution. Par contre, étant donné leurs gros temps d’exécution, cette instabilité est plus préjudiciable aux deux autres algorithmes.\\

Tous les algorithme sont par contre stable du point de vue de la qualité de solution même si 2-opt varie un peu plus cela reste petit, et cela deviens négligeable pour les deux autres.

\section{Path-relinking}

Nous avons expérimenté deux manières différentes d'implémenter un path-relinking. Le premier, le plus naturel, consiste à inverser l'emplacement de deux villes. De cette manière on conserve l'admissibilité de la solution. Le second, plus arbitraire, consiste à remplacer certaines villes par une autre. Cette dernière solution recquiert une réparation régulière.

\subsection{Mouvement swap}

Ce path-relinking a pour voisinage toutes les permutations possibles de deux villes. On a donc un voisinage en $O(N^2)$. Dans sa première version, l'algorithme parcourt la liste des permutations et l'effectue si celle-ci n'éloigne pas la solution courante de la solution objectif. Un tel sous-ensemble de voisinage permet de ne jamais s'éloigner de la fonction objectif tout en offrant une grande liberté de mouvement afin d'explorer de nombreuses solutions.\\

Pour la deuxième version de cet algorithme, nous nous sommes inspirés d'un travail de Mauricio Resende et de Paola Festa. Dans cette version, nous parcourons tous les voisins possibles et nous effectuons celui qui améliore le mieux la solution tout en nous approchant d'au moins une ville de la solution objectif.\\

De plus, comme nous listons toutes les permutations possibles, nous nous permettons d'enregistrer une solution sans effectuer le mouvement si la solution améliore la fonction objectif.\\

Après expérimentations, nous avons constaté que le path-relinking ralentit de manière excessive le GRASP. Utiliser comme voisinage la liste des permutations consécutives ne garantit pas la convergence de l'algorithme, nous avons donc décider d'accélerer la convergence de l'algorithme. En effet, la dernière version de notre algorithme n'effectue un mouvement que si celui-ci permet d'approcher la fonction objectif avec les deux villes. Ce choix a permis de beaucoup accélérer le path-relinking en affectant peu son efficacité.\\

Cet algorithme est donc en $O(N^2)$ sans prendre en compte la vitesse de convergence, nécessitant au maximum $N/2$ swaps, nous avons donc une complexité en $O(N^3/2)$, soit $O(N^3)$.\\

\subsection{Mouvement insertion}

Ce path-relinking a pour voisinage toutes les insertions possibles d'une ville à la place d'une autre de telle manière que la nouvelle ville à l'emplacement i est celle de la ville objectif à l'emplacement i. Nous avons donc un voisinage en $O(N)$. Ce type de mouvement ne garantit pas l'admissibilité de la solution, ainsi, un algorithme glouton replace les villes manquantes (ou doublées) lorsqu'il y a un certain nombre nb de villes absentes de la solution (nb étant un paramètre de la fonction).\\

Nous avons aussi tenté une version dans laquelle nous choisissons l'insertion améliorant le plus la solution. Cet algorithme cet avéré peu efficace, même s'il réussit parfois à trouver une meilleure solution. Nous avons donc préféré le path-relinking avec swaps.\\

Cet algorithme a donc une complexité de $O(N^)$ sans considérer la vitesse de convergence ni la réparation. Il y a au plus $N$ itérations, nous obtenons donc une complexité de $O(N^2)$ plus la réparation dont le coût dépend du paramètre donné (une valeur plus élevée permet des réparations plus longues mais moins régulières et inversement). Le coût de la réparation étant en $O(N*nb)$ avec $nb<N/2$, la réparation n'affecte pas la complexité algorithmique totale.\\

\subsection{Inclusion dans GRASP}

On commence par tester le path-relinking avec la version de GRASP qui pourrais le plus en bénéficier : 2-opt à arrêt probabiliste à 2\%. En effet cette version est rapide mais elle soufre d'un manque de qualité de ses solutions. Donc un ralentissement peu être acceptable et de meilleures solutions corrigerai son défaut.

\begin{table}[!h]
\centering
\begin{tabular}{|*{4}{c|}}
\hline
nbVille & en plus profonde descente & en descente & sans path-relinking \\
\hline
48 & 215 & 215 & 216 \\
52 & 1.33 & 1.09 & 1.34 \\
130 & 2.80 & 2.87 & 2.87 \\
150 & 3.34 & 3.34 & 3.38 \\
280 & 7.33 & 7.16 & 7.33 \\
\hline
\end{tabular}
\caption{différence de valeur de 2-opt probabiliste en fonction du path-relinking}
\label{val2optgraspPR}
\end{table}

\begin{table}[!h]
\centering
\begin{tabular}{|*{4}{c|}}
\hline
nbVille & en plus profonde descente & en descente & sans path-relinking \\
\hline
48 & 0.2546 & 0.1317 & 0.1599 \\
52 & 0.0390 & 0.0369 & 0.0250 \\
130 & 0.2683 & 0.2410 & 0.2038 \\
150 & 0.3009 & 0.1947 & 0.1390 \\
280 & 0.5504 & 0.6139 & 0.4688 \\
\hline
\end{tabular}
\caption{temps d’exécution de 2-opt probabiliste en fonction du path-relinking}
\label{temps2optgraspPR}
\end{table}

Donc le path-relinking n’améliore que très peu la qualité des solutions mais il ne ralentit pratiquement l'algorithme.\\

Dans cet optique de payer un peu plus de temps pour avoir de meilleures solutions le test a été fait avec 2-opt en plus profonde descente, toujours en arrêt probabiliste à 2\%. Dans cette configuration la meilleure solution initiale généré par NNH est amélioré par 2-opt PPD.\\

\begin{table}[!h]
\centering
\begin{tabular}{|*{4}{c|}}
\hline
nbVille & en plus profonde descente & en descente & sans path-relinking \\
\hline
48 & 215 & 215 & 215 \\
52 & 0.4947 & 0.5664 & 0.5664 \\
130 & 2.16 & 2.15 & 2.22 \\
150 & 1.27 & 1.28 & 1.29 \\
280 & 6.00 & 6.00 & 6.00 \\
\hline
\end{tabular}
\caption{différence de valeur de 2-opt PPD probabiliste en fonction du path-relinking}
\label{val2optPPDgraspPR}
\end{table}

\begin{table}[!h]
\centering
\begin{tabular}{|*{4}{c|}}
\hline
nbVille & en plus profonde descente & en descente & sans path-relinking \\
\hline
48 & 0.3901 & 0.392 & 0.3715 \\
52 & 0.1152 & 0.109 & 0.1143 \\
130 & 2.105 & 2.595 & 2.649 \\
150 & 1.506 & 1.285 & 1.311 \\
280 & 16.47 & 16.47 & 16.55 \\
\hline
\end{tabular}
\caption{temps d’exécution de 2-opt PPD probabiliste en fonction du path-relinking}
\label{temps2optPPDgraspPR}
\end{table}

On a donc une nouvelle configuration intéressante de GRASP : 2-opt PPD avec l’arrêt probabiliste à 2\% et un path-relinking en plus profonde descente. En effet dans cette configuration l'algorithme se termine en un temps respectable et trouve des solutions de bonne qualité. Et on notera que le path-relinking permet de faire ressortir l’arrêt probabiliste car les bonnes solutions trouvé par ce dernier permettent d’arrêter plus vite l'algorithme. De ce fait le coût supplémentaire du path-relinking se compense tellement bien de cette manière que l'algorithme est parfois plus rapide avec.\\

L'ajout du path-relinking ralentit peu GRASP mais il le fait tout de même donc la possibilité de l'ajouter aux autres configuration à été rejeté car ces dernières sont déjà très lentes.\\

\section*{Conclusion}
Bien que pertinente, nous remarquons que notre heuristique de construction R.G.S.C. donne globalement une solution moins bonne que celle du N.N.H. En effet, bien le R.G.S.C. effectue plus de comparaisons celui-ci fournit des chemins plus long d'au moins 10\%. De plus, celui-ci donne une solution de départ plus difficile à améliorer avec les algorithmes que nous avons implémentés. Pour certaines instances, l'amélioration n'a pas pu se terminer en un temps raisonnable et a été stopée.\\
Cet algorithme est donc moins bon qu'il n'en avait l'air. Cela dit, nous allons réfléchir à d'éventuelles améliorations basées sur le même principe.\\
La plupart des difficultés rencontrées se situe dans l'algorithmie. En effet, estimer en permanence la taille de la solution après modification par le 3-opt était une tâche plus difficile qu'il n'en parraîssait. Ou encore, les conditions d'arrêt du VNS. Ces deux aspects sont donc à travailler et nous espérons pouvoir les améliorer avant un prochain rendu de notre travail. Une étude des instances et peut être pertinente afin d'adapter nos algorithmes d'amélioration aux caractéristiques d'un chemin fournie par une heuristique de construction pour une instance donnée.\\
Ces aspects sont importants pour réduire l'écart de temps d’exécution entre les algorithmes VNS et VND. En effet, cet écart est élevé et peut encore être amélioré d'après nos tests de performance.\\

TODO : refaire la conclusion car beaucoup a été fait depuis cette version\\

Ouverture pour ma partie : étudier plus finement l'impact du choix de la première ville dans NNH, quitte à la choisir aléatoirement dans GRASP pour augmenter la diversité.\\

conclusion de GRASP : plusieurs configurations intéressantes. Selon le temps don on dispose on peut avoir de plus ou moins bonnes solutions (par ordre croissant de temps dispo) :
\begin{itemize}
\item 2-opt à 2\% path-relinking en descente et init par 2-opt
\item 2-opt PPD à 2\% path-relinking en plus profonde descente et init par 2opt PPD
\item vnd avec nogood de nbVille/5
\item vns-2 avec nogood de bnVille/5, dans l'absolu on peut y adjoindre le path-relinking pour tenter d'avoir de meilleures solutions. Mais on renforce alors encore plus son coté extrême.
\end{itemize}
\end{document}