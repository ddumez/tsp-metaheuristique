\documentclass[12pt,a4paper]{article}
\usepackage[utf8]{inputenc}
\usepackage[toc,page]{appendix}
\usepackage{amsmath}
\usepackage{amsthm}
\usepackage{amsfonts}
\usepackage{amssymb}
\usepackage[overload]{empheq}
\usepackage{color}
\usepackage[dvipsnames]{xcolor}
\usepackage{fullpage}

\title{Rapport intermédiaire du projet de comparaison de méthodes de résolution approché du T.S.P.}
\author{Dorian Dumez \and Jocelin Cailloux}

\begin{document}
\maketitle
\section{Heuristiques de constructions}
\subsection{N.N.H.}
C'est une heuristique de construction gloutonne qui, partant d'une ville donné, va construire le cycle hamiltonien en choisissant comme prochaine ville celle qui se situe le plus proche de la vile actuelle.\\
On vois tout de suite l'impact du choix de la ville de départ sur la solution. Mais estimer la qualité de la solution en fonction de chaque ville est compliqué donc elle est choisie de manière arbitraire. En effet on prend toujours comme point de départ la première ville situé dans le tableau de notre distancier.\\
Notre distancier représentant le graphe sous forme d'une matrice d'adjacence on dispose toujours de la distance entre deux ville en temps constant. Cet algorithme va, pour chaque ville (on les parcours toutes une unique fois étant donne que l'on construit un cycle hamiltonien), parcourir toutes les villes non déjà parcourues pour trouver la plus proche. On en déduit que cet algorithme à une complexité temporelle en $O(nbVille^2)$. De plus sa complexité spatiale est en $O(nbVille)$, en effet on ne fait qu’allouer le parcours correspondant à la permutation identité avant de le modifier sur place.\\

\subsection{R.G.S.C}

\section{Opérateurs de modification}
\subsection{2-opt}
Cet opérateur consiste à sélectionner deux arc (i.e. deux paire de ville adjacente dans le parcours) et a tester si leur croisement fournit une meilleure solution. Si on pose iD, iF les extrémités du premier arc et jD, jF celle du deuxième arc, on entend par croiser les arc aller de iD à jD puis de iF à jF au lieu de iD à iF puis de jD à jF comme c’était le cas dans la solution. Graphiquement on observe que c'est en réalité l'inverse qui améliore la solution, en effet si les deux arc s’intersectent dans le plan (d'une carte où les villes sont positionnées dans un repère par exemple) alors les inverser fait disparaître cette intersection et réduit la taille du chemin parcouru.\\
Il est alors évident que les 4 villes (iD, iF, jD, jF) doivent êtres différentes car sinon on ne fait que modifier l'ordre de parcours de ces villes (on ne fait qu'inverser une permutation) mais le T.S.P. sur lequel on travaille est symétrique donc cela ne change rien. Il est aussi évident que l'on considère des paires d'arc et non des couples car cela reviendrai à tester deux fois le même changement.\\

Nous avons implémenté cet opérateur sous deux formes : en descente et en plus profonde descente.\\
En plus profonde descente on teste toute les paires d'arc et on applique seulement celle qui conduit à la meilleure solution, ici celle dont le parcours est le plus cour.\\
En descente on teste aussi toutes le paires d'arc mais cette fois ci on applique la modification dès qu'elle amélioré la solution. On ne s’arrête pas à la première solution trouve pour éviter de s'appesantir sur le début du cycle et pour éviter de le re-tester un trop grand nombre de fois à partir du moment où il se serait stabilisé.\\
Dans les deux cas on va tester $O(nbVille^2)$ changement mais appliquer l'opérateur 2-opt s’effectue en $O(nbVille)$ car on doit inverser le sens de parcours des villes situé entre iF et jD. Donc l'algorithme de descente est en $O(nbVille^3$ tandis que celui de plus profonde descente est en $O(nbVille^2)$. Mais cette complexité n'est que celle de amelioreSol2opt et amelioreSol2optPPD, donc ne prend pas en compte la vitesse de convergence de la solution vers un minimum local.\\
Ces complexité sont permise par le fait que l'on peut connaître à l'avance et en temps constant la taille du cycle hamiltonien après la modification.
 
\subsection{3-opt}
Cet opérateur est une généralisation du 2-opt avec 3 arc. Tout les mouvements permis par le 2-opt le sont donc aussi par le 3-opt. Donc contrairement au 2-opt qui n'offre qu'une seule possibilité de modification, le 3-opt en offre 7 don certaines ne peuvent pas être réalisé par une succession de 2-opt.\\

Cet opérateur a aussi été implémenté en descente et en plus profonde descente.\\
Pour les même raison que pour le 2-opt, dans l'algorithme de descente on teste toutes les triplet possible à chaque fois au lieu de s’arrêter au premier mouvement améliorant trouvé.\\
Mais cette fois ci je n'ai pas réussis à trouver une méthode me permettant de connaître à l'avance la longueur du cycle hamiltonien après la modification. Donc pour tester une modification la solution est applique puis la modification appliqué avant de calculer la longueur du cycle obtenu. Donc chaque test se fait en $O(nbVille)$.\\
La complexité du 3-opt est donc de $O(nbVille^4)$ car il y a $O(nbVille^3)$ ensemble de 3 arc possible et le test et la modification se font de en $O(nbVille)$, la modification restant en $O(nbVille)$ car des inversions du sens de parcours sont nécessaire sur la plupart d'entre elles.\\
Et avec le 3-opt la plus profonde descente à la même complexité puisque on est obligé d'appliquer toutes les modifications pour les tester.\\

\section{Expérimentation}
\subsection{En partant de N.N.H.}
PPD designe la variante de l'algorithme utilisant la plus profonde descente.\\

\begin{table}[h]
\centering
\begin{tabular}{|*{10}{c|}}
  \hline
  nbVille & N.N.H. & 2-opt & 3-opt & 2-opt PPD & 3-opt PPD & vnd & vnd PPD & vns & vns PPD \\
  \hline
  48 & 281 & 225 & 230 & 223 & 225 & 225 & 223 & 224 & 223 \\
  52 & 19 & 6,8 & 5,9 & 4 & 4 & 6,3 & 4 & 5,6 & 4 \\
  130 & 24 & 10 & 12 & 6 & 5,7 & 10 & 5,7 & 8,4 & 5,7 \\
  150 & 25 & 3,8 & 2,4 & 1,4 & 1,4 & 3,8 & 1,4 & 2,3 & 1,4 \\
  280 & 23,3 & 10,7 & 7,5 & 7,5 & 7,5 & 10 & 7,5 & 7,5 & 7,5 \\
  \hline
\end{tabular}
\caption{distance supplémentaire en pourcentage par rapport à la solution optimale}
\label{NNHpourcentage}
\end{table}

\begin{table}[h]
\centering
\begin{tabular}{|*{10}{c|}}
  \hline
  nbVille & N.N.H. & 2-opt & 3-opt & 2-opt PPD & 3-opt PPD & vnd & vnd PPD & vns & vns PPD \\
  \hline
  48 & 1.2e-05 & 0.00026 & 0.375 & 0.0007 & 1.6701 & 0.166 & 0.173 & 0.664 & 0.33 \\
  52 & 2.2e-05 & 0.00026 & 0.474 & 0.00097 & 2.7708 & 0.464 & 0.235 & 0.462 & 2.77 \\
  130 & 9.5e-05 & 0.00354 & 18.5 & 0.01105 & 206.4 & 18.7 & 18.5 & 37.1 & 66.7 \\
  150 & 8.9e-05 & 0.0026 & 33.5 & 0.01409 & 403.6 & 17.4 & 16.9 & 69.6 & 68.9 \\
  280 & 0.00069 & 0.02320 & 616 & 0.0879 & 7785.99 & 671.396 & 213.633 & 440.933 & 843.438 \\ 
  \hline
\end{tabular}
\caption{temps d’exécutions en secondes}
\label{NNHpourcentage}
\end{table}

\subsection{En partant de R.G.S.C}

\end{document}
